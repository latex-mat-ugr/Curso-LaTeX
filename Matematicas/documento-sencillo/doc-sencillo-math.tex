\documentclass[11pt]{article}

\usepackage[T1]{fontenc} % Codificación de salida
\usepackage[a4paper,margin=3cm]{geometry} % Márgenes y tamaño de página

% Colores
\usepackage[table,svgnames,dvipsnames,x11names]{xcolor} % Colores
%% colores definidos por el usuario
\definecolor{funcion1}{rgb}{0,0,0.6}
\definecolor{funcion2}{rgb}{0.6,0,0}
\definecolor{funcion3}{rgb}{0,0.6,0}
\definecolor{punto}{rgb}{0.49,0.49,1}
\definecolor{relleno}{rgb}{0.6,0.2,0}

% gráficos
\usepackage{graphicx} % inclusión de gráficos
\usepackage{wrapfig}  % texto que se adapta a una figura

\usepackage{pgfplots,tikz}     % gráficos con pgfplots y tikz
\pgfplotsset{compat=1.18}     % versión de pgfplots
\graphicspath{{tikz/},{fig/}}  % directorios de gráficos
\usepackage[most]{tcolorbox}   % cajas de colores

% paquetes relacionados con matemáticas
\usepackage{amsmath,amssymb,amsthm}  % matemáticas AMS
\usepackage[showonlyrefs]{mathtools} % mejoras y correcciones para matemáticas

\numberwithin{equation}{section} % numeración de ecuaciones por sección
\allowdisplaybreaks[2] % permite dividir ecuaciones largas en varias páginas

\DeclareMathOperator{\deter}{determinante}
\DeclareMathOperator{\dist}{dist}

% paquetes relacionados con matemáticas (opcionales)
\usepackage{cancel} 			% tachar en una ecuación
\usepackage{annotate-equations}     % Añadir explicaciones en ecuaciones
\usepackage{tasks} 			% Listas
\usepackage{booktabs}			% Tablas
\usepackage{tabularray}			% tablas y arrays
\usepackage{witharrows}			% flechas entre ecuaciones

% paquetes relacionados con texto
\usepackage{enumitem} % personalizar listas (véase también enumerate y tasks) 
\usepackage{multicol} % texto en varias columnas
\usepackage{siunitx} % escribir unidades del SI

% paquetes relacionados con química
\usepackage{chemfig} % fórmulas químicas
\usepackage[version=4]{mhchem} % ecuaciones químicas

% código
\usepackage{listings}
\usepackage{minted}
\usepackage{markdown}

% idioma
\usepackage[spanish, es-lcroman, es-tabla,es-noshorthands]{babel}   % el último idioma es el principal, no hace falta ponerlo en main
\decimalpoint % Cambia la coma por el punto en el modo matemático
\unaccentedoperators % Desactiva el uso de acentos en los operadores matemáticos

% entrecomillados
\usepackage[babel=true,autostyle=true,spanish=spanish]{csquotes}

% Fuente de texto
\usepackage{newpxtext}        % Fuente de texto
\usepackage{newpxmath}        % Fuente matemática
\usepackage{fontawesome5}     % Iconos
\usepackage[final,babel]{microtype} % Mejoras tipográficas

% teoremas, proposiciones y definiciones
\theoremstyle{plain}                            % Estilo de los teoremas
\newtheorem{teorema}{Teorema}[section]
\newtheorem{proposicion}[teorema]{Proposición}
\newtheorem{lema}[teorema]{Lema}
\newtheorem{corolario}[teorema]{Corolario}
\theoremstyle{definition}                       % Estilo de las definiciones
\newtheorem{definicion}[teorema]{Definición}
\newtheorem{ejemplo}[teorema]{Ejemplo}
\theoremstyle{remark}                           % Estilo de las observaciones
\newtheorem{remark}[teorema]{Observación}

% Comandos propios

\newcommand{\abs}[1]{\ensuremath\left\vert #1 \right\vert}
\newcommand{\norm}[1]{\ensuremath\left\Vert #1 \right\Vert}

% Metadatos del documento
\author{El Autor}
\title{El título}

% Cuerpo del documento
\begin{document}

\maketitle

Este fichero es un ejemplo \enquote{sencillo} (``sencillo'') de documento en \LaTeX{} en el que se muestra el uso de algunas herramientas y paquetes útiles para la escritura de documentos matemáticos. 

%Como decía George Bernad Shaw \foreigntextquote{english}{Both optimists and pessimists contribute to society. The optimist invents the aeroplane, the pessimist the parachute.} 

\section{Definición}

Sea $I$ un intervalo y sean $a$, $b \in I$ con $a<b$. Si $x\in [a,b]$, $x$ se puede escribir como \emph{combinación convexa} de $a$ y $b$:
\[
      x= \frac{b-x}{b-a}\cdot a + \frac{x-a}{b-a}\, b.
\]
Otras fórmulas
\[
      \lim_{t \to 0} \frac{f(a+t,b)-f(a,b)}{t}= \frac{\partial f}{\partial x}(a,b).
\]



\begin{definicion}\label{def:convexa}
      Sea $I$ un intervalo y sea $f \colon I \rightarrow \mathbb{R}$.
      \begin{enumerate}
            \item La función $f$ es \emph{convexa} si para cualesquiera $a$, $b \in I$ con $a<b$ y cualquier $t \in [0,1]$ se cumple que
                  \[
                        f\left((1-t)a+tb\right) \leq (1-t)f(a)+tf(b).
                  \]
            \item La función $f$ es \emph{cóncava} (resp. estrictamente cóncava) si $-f$ es convexa (resp. estrictamente cóncava).
      \end{enumerate}
\end{definicion}

\begin{ejemplo}
      La función $f(x)=x^2$ es convexa en todo $\mathbb{R}$. En efecto, si $a$, $b \in \mathbb{R}$ y $t \in [0,1]$,
      \begin{align*}
            f\left( (1-t)a+tb \right) & = (1-t)^2 a^{2}+t^{2}b^{2} +2t(1-t)ab, \\
            (1-t)f(a)+tf(b)           & = (1-t)a^2+tb^2.
      \end{align*}
      Para ver cuál es mayor, calculamos la diferencia:
      \begin{align*}
            (1-t)f(a) & +tf(b)  - \bigl( f\left( (1-t)a+tb \right) \bigr) \\
            \begin{split}
                   & = (1-t)a^2+tb^2 - \left( (1-t)^2 a^{2}+t^{2}b^{2} +2t(1-t)ab \right) \\
                   & = t(1-t) \left( a-b \right)^2 \geq 0.
            \end{split}
      \end{align*}
\end{ejemplo}

El paquete \texttt{witharrows} añade dos entornos: \texttt{WithArrows} y \texttt{DispWithArrows} que funcionan como \texttt{aligned} y \texttt{align}:
\begin{DispWithArrows*}[tikz=blue]
      A & = \bigl((a+b)+1\bigr)^2 \Arrow[jump=2]{desarrollamos} \\
      & = (a+b)^2 + 2(a+b) +1 \\
      & = a^2 + 2ab + b^2 + 2a + 2b +1
\end{DispWithArrows*}

\subsection{El conjunto de las funciones convexas}

Veamos algunas propiedades del conjunto de las funciones convexas (ver definición~\ref{def:convexa}).

\begin{proposicion}
      Sean $f$, $g \colon I \rightarrow \mathbb{R}$ dos funciones convexas y
      sea $\alpha>0$. Entonces
      \begin{enumerate}
            \item $f+g$ es convexa,
            \item $\alpha f$ es convexa, y
            \item $\max \{f ,g\}$ es convexa.
      \end{enumerate}
\end{proposicion}
\begin{proof}
      \begin{enumerate}
            \item Sean $a$, $b\in I$, y sea $t\in [0,1]$, entonces
                  \begin{equation}
                        \begin{split}
                              (f+g) \left( (1-t)a+tb \right) & = f \left( (1-t)a+tb \right) + g \left( (1-t)a+tb \right) \\
                                                             & \leq (1-t)f(a)+tf(b)+(1-t)g(a)+t g(b)                     \\
                                                             & = (1-t)(f+g)(a) + t(f+g)(b).
                        \end{split}
                  \end{equation}
            \item Sean $a$, $b\in I$, y sea $t\in [0,1]$, entonces
                  \begin{equation}
                        \begin{aligned}
                              (\alpha f) \left( (1-t)a+tb \right) & = \alpha f \left( (1-t)a+tb \right)     \\
                                                                  & \leq \alpha (1-t)f(a)+ \alpha t f(b)    \\
                                                                  & = (1-t)(\alpha f)(a) + t (\alpha f)(b).
                        \end{aligned}
                  \end{equation}
            \item Sean $a$, $b\in I$, y sea $t\in [0,1]$, entonces
                  \[
                        \begin{aligned}
                              \max \{ f,g\} \left( (1-t)a+tb \right) & = \max \left\{ f\left( (1-t)a+tb) \right),g \left((1-t)a+tb\right) \right\} \\
                                                                     & \leq \max \left\{ (1-t)f(a)+tf(b),(1-t)g(a)+t g(b) \right\}                 \\
                                                                     & = (1-t) \max\{f,g\}(a) + t \max\{ f,g\} (b). \qedhere
                        \end{aligned}
                  \]
      \end{enumerate}
\end{proof}


\section{Segunda sección}


\begin{teorema}[O. Stolz] \label{th:cc}
      Sea $I$ un intervalo abierto y sea $f \colon I \rightarrow$ una función convexa.
      \begin{enumerate}
            \item La función $f$ tiene derivadas laterales en todos los puntos. En particular, $f$ es continua.
            \item $f'_{+}$ y $f'_{-}$ son funciones crecientes.
      \end{enumerate}
\end{teorema}

\section{Tercera sección}

\begin{teorema}[de la función inversa]\label{th:inversa}
      Sean $A\subset \mathbb{R}^N$ un abierto y $f\colon A\rightarrow \mathbb{R}^{N}$ un campo vectorial de clase
      $\mathcal{C}^1$. Supongamos que existe $a\in A$ tal que
      \[
            \deter \bigl (J_f(a)\bigr )=\det
            \left(
            \begin{matrix}
                        D_1 f_1 (a) & \dotsc & D_{N} f_{1}(a) \\
                        \hdotsfor[2]{3}                       \\
                        D_1f_N(a)   & \dotsc & D_N f_N(a)
                  \end{matrix}
            \right)
            \neq 0.
      \]
      Entonces existe un entorno abierto $U$ de $a$ contenido en $A$ tal que $V:=f(U)$ es un abierto de $\mathbb{R}^N$, $f$ es un difeomorfismo de clase $\mathcal{C}^1$ de $U$ sobre $V$, y para cada $x\in U$ se verifica
      \[
            J_{f^{-1}}\bigl (f(x)\bigr )=
            \begin{pmatrix}
                  D_1f_1(x) & \dotsc & D_Nf_1(x) \\
                  \hdotsfor[2]{3}                \\
                  D_1f_N(x) & \dotsc & D_Nf_N(x)
            \end{pmatrix}^{-1}.
      \]
\end{teorema}

\begin{teorema}[de la convergencia monótona]
      Sea $(\Omega,\mathcal{A},\mu)$ un espacio de medida. Si $\{f_n\}$ es una sucesión monótona de funciones integrables en $\Omega$  verificando que la sucesión de sus integrales $\big
            \{\int_{\Omega}f_n\  d\mu \big \}$ está acotada, entonces $\{f_n\}$ converge $\text{c.p.d.}$ a una función $f$ integrable en $\Omega$ y
      \[
            \int_{\Omega}f\, \mathrm{d}\mu=\lim \int_{\Omega}f_n\, \mathrm{d}\mu .
      \]
\end{teorema}

\section{Miscelánea}

\subsection{Unidades}

El paquete \texttt{siunitx} permite escribir unidades del Sistema Internacional de forma sencilla. Por ejemplo, la velocidad de la luz en el vacío es \SI{300000}{\km\per\second}. Otra forma de escribirlo es \SI{3e5}{\km\per\second}. Con \texttt{num} podemos formatar los números, por ejemplo 
\[
      \begin{split}
      \pi & \approx\num{3.141592653589793} \\
      e^{15} & \approx \num{3269017.3724721107} \approx \num[scientific-notation=true]{3269017.3724721107}
      \end{split}
\]      

Además de escribir unidades, \texttt{siunitx} permite escribir ángulos, por ejemplo, $\ang{30;15;20}$ o $\ang{30.25}$, números complejos \complexnum{2+3i} en forma cartesiana o polar \complexnum{3:60}. Además de esto añade un tipo nueva de columna en los entornos tabulados para alinear números, parte entera y decimal.

\subsection{Física y Química}

Algunos paquetes para Física:
      \begin{enumerate}
            \item revtex
            \item tikz-feynman
            \item siunitx
            \item struct
      \end{enumerate}

Algunos paquetes para Química:
      \begin{enumerate}
            \item chemfig
            \item mhchem
            \item chemformula
            \item chemmacros
            \item substances
            \item elements
            \item carbohydrates
      \end{enumerate}      

Hay muchos paquetes que simplifican la escritura de fórmulas químicas. Por ejemplo, con \texttt{chemfig} se pueden escribir fórmulas químicas de forma sencilla. Por ejemplo, la molécula de agua se puede escribir como \chemfig{H_2O}.

\begin{equation}
      \ce{2H2 + O2 -> 2H2O} 
\end{equation}
\begin{equation}
      \ce{N2 + 3H2 -> 2NH3}
\end{equation}

\begin{table}[h!]
      \centering
      \begin{tabular}{|c|c|}
            \hline
            Fórmula Química & Nombre            \\
            \hline
            \ce{H2O}        & Water molecule    \\
            \ce{NaCl}       & Sodium chloride   \\
            \ce{C6H12O6}    & Glucose           \\
            \ce{^{14}C}     & Carbon-14 isotope \\
            \ce{Na^+}       & Sodium ion        \\
            \ce{H2O_{2}}    & Hydrogen Peroxide \\
            \hline
      \end{tabular}
      \caption{Tabla de fórmulas químicas y sus nombres}
      \label{tab:quimica}
\end{table}

\subsection{Código}

\subsubsection{Ejemplo de código en Python}

\begin{minted}{python}
# Ejemplo de código en Python
def suma(a, b):
return a + b
resultado = suma(3, 5)
print("El resultado de la suma es: {resultado}")
\end{minted}

\subsubsection{Ejemplo de código en \LaTeX{}}

\begin{minted}{latex}
      \begin{table}[h!]
            \centering
            \begin{tabular}{|c|c|}
                  \hline
                  Fórmula Química & Nombre            \\
                  \hline
                  \ce{H2O}        & Water molecule    \\
                  \ce{NaCl}       & Sodium chloride   \\
                  \ce{C6H12O6}    & Glucose           \\
                  \ce{^{14}C}     & Carbon-14 isotope \\
                  \ce{Na^+}       & Sodium ion        \\
                  \ce{H2O_{2}}    & Hydrogen Peroxide \\
                  \hline
            \end{tabular}
            \caption{Tabla de fórmulas químicas y sus nombres}
            \label{tab:quimica}
      \end{table}      
\end{minted}



\end{document}
