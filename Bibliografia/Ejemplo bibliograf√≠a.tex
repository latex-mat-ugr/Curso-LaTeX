\documentclass{article}

\usepackage[spanish]{babel}

\begin{document}
\title{Ejemplo de uso de bibliografía con \texttt{bibtex}}
\maketitle

Se ha creado un documento \texttt{referencias.bib} con la base de datos bibliográfica con dos elementos y se ha cargado con \texttt{\textbackslash bibliography\{referencias\}}. En la lista de referencias sólo aparecen aquellos elementos que han sido citados con el comando \texttt{\textbackslash cite}.


Ejemplo de referencia: \emph{Como podemos ver en \cite{Euler1984}}.
% Las revistas suelen proporcionar un estilo predefinido para sus plantillas a través de un fichero `.bst`. En la instalación habitual de LaTeX tenemos a nuestra disposición diferentes estilos bibliográficos.

\bibliographystyle{plain} % plain, alpha, amsalpha, apalike, abbr
\bibliography{referencias}

\end{document}
