\chapter{Algunos entornos}

%el primer argumento es el nombre del entorno, el segundo el número de parámetros que le pasamos, el tercero indica cómo empieza el entorno, y el último cómo termina (en este caso terminamos con un cuadrado hueco)
\newenvironment{ejercicio}[1]{\textbf{Ejercicio número #1}}{\qed}

Veamos cómo escribir un ejercicio.

\begin{ejercicio}{1}
Escribe esto con otras palabras.
\end{ejercicio}

Otra forma con contadores.\index{contadores}
%definimos un contador, así no tenemos que poner manualmente número a los ejercicios
\newcounter{ejer_num}
%ponemos el contador a 1
\setcounter{ejer_num}{1}
%definimos ahora un nuevo entorno, sin parámetros; el contador se incrementa con \stepcounter{}
\newenvironment{ejer}{\textbf{Ejercicio \arabic{ejer_num}: \stepcounter{ejer_num}}\begin{itshape}}{\end{itshape}}

\begin{ejer}
Una de melón.
\end{ejer}

\begin{ejer}
Otra de sandía.
\end{ejer}

\section{Otros entornos}

%el entorno teorema irá numerado usando las secciones con el formato (nºsección).(nºteorema en la sección)
\newtheorem{teorema}{Teorema}[section]
%con esta forma de definir nota, hacemos que tenga el mismo tipo de contador que el entorno teorema anteriormente definido
\newtheorem{nota}[teorema]{Aclaración}

\begin{teorema}\label{tonto}
Las ranas son verdes.
\end{teorema}
%el entorno proof está predefinido en amsart, al usar el paquete babel con la opción spanish, imprimirá "Demostración:" en vez de "Proof:".
\begin{proof}
Así lo decía Aristóteles, y nosotros no vamos a llevarle la contraria.
\end{proof}

\begin{nota}
Alguien probó que el Teorema \ref{tonto} es falso, pues encontró una rana marrón.
\end{nota}

