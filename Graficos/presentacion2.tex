% Modelo de slides para projetos de disciplinas do Abel
\documentclass[10pt]{beamer}

\usetheme[progressbar=frametitle]{metropolis}
\usepackage{appendixnumberbeamer}
\usepackage[numbers,sort&compress]{natbib}
\bibliographystyle{plainnat}
\usepackage{booktabs}
\usepackage[scale=2]{ccicons}
\usepackage{xspace}
\newcommand{\themename}{\textbf{\textsc{metropolis}}\xspace}
%%%%%%%%%%%%%%%%%%%%%%%%%%%%%%%%%%%%%%%%%%%%
\usepackage[spanish]{babel}
% Paquete pdfpages incluye páginas completas de ficheros pdfs
\usepackage{pdfpages}
%TikZ 
\usepackage{pgf,tikz}
\usetikzlibrary{babel,shapes,arrows}
% PGFPlots
\usepackage{pgfplots}
\pgfplotsset{width=6cm,compat=1.12} %{compat=yourversion}
%paquete chemfig: gráficos de química
\usepackage{chemfig}
%paquete Wrapfig
\usepackage{wrapfig}

%%%%%%%%%%%%%%%%%%%%%%%%%%%%%%%
\title{Creación de gráficos con \LaTeX{}}
% \subtitle{Subtítulo}
% \date{\today}
\date{}
\author{Óscar Sánchez Romero}
\institute{Dpto. Matemática Aplicada, UGR}
% \titlegraphic{\hfill\includegraphics[height=1.5cm]{logo.pdf}}
%%%%%%%%%%%%%%%%%%%%%%%%%%%%%%%%%%%
\begin{document}

\maketitle

\begin{frame}{Contenidos}
  \setbeamertemplate{section in toc}[sections numbered]
  \tableofcontents[hideallsubsections]
\end{frame}

\section{Introducción}


%%%%%%%%%%%%%%%%%%%%%%%%%%%%%%%%%%%%%%%%%%%%%
\begin{frame}{Gr\'aficos insertados vs generados}
%Cada vez recurrimos m\'as a representaciones gr\'aficas para ejemplificar. 

La inclusi\'on de muchos documentos gr\'aficos en un mismo documento \LaTeX $ $ tienen varios inconvenientes:
\begin{itemize}
\item Dan como resultado documentos muy pesados.
\item Pese a ello, la calidad de los gr\'aficos insertados no siempre es 
 \'optima.
 \end{itemize}

La soluci\'on que \LaTeX \ adopt\'o   hace tiempo es algo que est\'a hoy d\'ia muy de moda:
\begin{center}
Do it yourself (DIY)
\end{center}
%Esto es,  el propio \LaTeX \  interpreta una serie de instrucciones y genera el 
%gr\'afico.  Otra cosa distinta es c\'omo generamos dicho c\'odigo...
\begin{itemize}
\item {\small \color{red} Ventajas: }Alta calidad y ficheros con peso reducido.\\
\item {\small \color{red} Inconveniente:} inversión de tiempo de aprendizaje.
\end{itemize}
%por ejemplo este documento con $31$ gr\'aficas ocupa 438 KB!!!. }
\end{frame}


%%%%%%%%%%%%%%%%%%%%%%%%%%%%%%%%%%%%%%%%%%%%%%%
\section{Creación de gráficos con LaTeX}
%%%%%%%%%%%%%%%%%%%%%%%%%%%%%%%%%%%%%%%%%

%%%%%%%%%%%%%%%%%%%%%%%%%%%%%%%%%%%%%%%%%
\begin{frame}[fragile]{Generaci\'on de gr\'aficos b\'asicos: entorno {\em picture}}

\LaTeX \  es capaz de realizar gr\'aficos sencillos 
directamente desde el entorno picture

\begin{verbatim}
\begin{picture}(100,7)
\put(2,1.5){\line(1,0){23.5}}  \put(13,0.5){$<$} 
\put(27,1.4){\circle*{1.5}}  \put(26.5,3){0}
\put(28.5,1.5){\line(1,0){22}} \put(39,0.5){$>$}
\put(52,1.4){\circle*{1.5}} \put(51.5,3){5}
\put(53.5,1.5){\line(1,0){23.5}} \put(63,0.5){$<$}
\end{picture}
\end{verbatim}


\begin{center} \setlength{\unitlength}{1mm}
\begin{picture}(100,7)
\put(2,1.5){\line(1,0){23.5}} 
\put(13,0.5){$<$} 
\put(27,1.4){\circle*{1.5}} 
\put(26.5,3){0}
\put(28.5,1.5){\line(1,0){22}}
\put(39,0.5){$>$}
\put(52,1.4){\circle*{1.5}}
 \put(51.5,3){5}
\put(53.5,1.5){\line(1,0){23.5}}
\put(63,0.5){$<$}
\end{picture}
\end{center}

Inconvenientes: es muy básico y limitado.
\end{frame}


%%%%%%%%%%%%%%%%%%%%%%%%%%%%%%%%%%%%%%%%%%
\begin{frame}[fragile]{Gr\'aficos con PSTricks y TikZ}
Tanto \href{http://www.ctan.org/pkg/pstricks}{\color{blue}PSTricks} como \href{http://www.texample.net/tikz/}{\color{blue}PGF-TikZ} son paquetes de LaTeX que permiten hacer casi cualquier cosa mediante un gran abanico de comandos espec\'ificos.

Podemos sacar provecho de ellos de v\'arias maneras:
\begin{enumerate}
\item Escribiendo nosotros directamente los c\'odigos (siempre que estemos dispuestos a invertir nuestro tiempo en ello). Hay disponibles numerosos manuales, y ejemplos:
\begin{itemize}
\item \href{http://www.texample.net/tikz/examples/}{\color{blue}http://www.texample.net/tikz/examples/}
\item \href{http://tug.org/PSTricks/main.cgi?file=examples}{\color{blue}http://tug.org/PSTricks/main.cgi?file=examples}
\end{itemize}
\item Empleando paquetes que facilitan su uso como PGFPlots.
\item Crear los gr\'aficos con otros programas y exportarlos a TikZ.
\end{enumerate}


\vspace{0.5cm}
{\small Observaci\'on:
Aunque PSTricks no es compatible con PDFLaTeX, existen versiones (spt-pdf o pdftricks)
que sí lo son.}
\end{frame}

\subsection{Tikz, primeros pasos}

%%%%%%%%%%%%%%%%%%%%%%%%%%%%%%%%%%%%%%%%%%
\begin{frame}[fragile]{Filosofía básica TikZ}
\begin{wrapfigure}{r}{4cm}
\begin{tikzpicture}[scale=1]
\draw[help lines] (0,0) grid (3,3);
\draw[->] (0,0) -- (2,2.5);
\end{tikzpicture}
%\caption{Gr\'afico sencillo TikZ}
\end{wrapfigure}
$ $
\begin{verbatim}
Preámbulo:
\usepackage{tikz} 
\usetikzlibrary{babel,shapes,arrows}

Cuerpo:
\begin{tikzpicture}
\draw[help lines]  (0,0) grid (3,3);
\draw[->]  (0,0) -- (2,2.5);
\end{tikzpicture}
\end{verbatim}
\vspace{1cm}
\href{https://cremeronline.com/LaTeX/minimaltikz.pdf}{\color{blue}A very minimal introduction to TikZ}\\
\href{https://es.wikibooks.org/wiki/Manual_de_LaTeX/Inclusi%C3%B3n_de_gr%C3%A1ficos/Gr%C3%A1ficos_con_TikZ}{\color{blue}Manual de LaTeX/Inclusión de gráficos/Gráficos con TikZ}
\end{frame}
%%%%%%%%%%%%%%%%%%%%%%%%%%%%%%%%%%%%%%%%%%
\begin{frame}[fragile]{Filosofía básica TikZ}

\begin{wrapfigure}{r}{4cm}
\begin{tikzpicture}
\draw[<->, line width = 1pt]  (3.5,0) --(0,0) -- (0,1.5);
\draw [blue, domain=0:pi]  plot (\x, {sin(\x r)*exp(\x/exp(2*pi))});
\end{tikzpicture}
\caption{Gr\'afico sencillo TikZ}
\end{wrapfigure}
$ $
\vspace{-1.5cm }
\begin{verbatim}
Preámbulo:
\usepackage{tikz} 
\usetikzlibrary{babel,shapes,arrows}

Cuerpo:
\begin{tikzpicture}
\draw[<->, line width = 1pt] 
(3.5,0) --(0,0) -- (0,1.5);
\draw [blue, domain=0:pi] 
plot (\x, {sin(\x r)*exp(\x/exp(2*pi))});
\end{tikzpicture}
\end{verbatim}
%\vspace{2cm}
%\href{https://cremeronline.com/LaTeX/minimaltikz.pdf}{\color{blue}A very minimal introduction to TikZ}\\
%\href{https://es.wikibooks.org/wiki/Manual_de_LaTeX/Inclusi%C3%B3n_de_gr%C3%A1ficos/Gr%C3%A1ficos_con_TikZ}{\color{blue}Manual de LaTeX/Inclusión de gráficos/Gráficos con TikZ}
\end{frame}



%%%%%%%%%%%%%%%%%%%%%%%%%%%%%%
\subsection{Uso de paquetes específicos}

%%%%%%%%%%%%%%%%%%%%%%%%%%%%%%%%%%%%%%%%%%%%%
\begin{frame}[fragile]{Generaci\'on de gr\'aficos en TikZ con PGFPlots}
Un primer ejemplo puede ser \href{http://pgfplots.net}{\color{blue}pgfplots} para representar funciones y datos.
\begin{wrapfigure}{r}{6cm}
	\begin{tikzpicture}
	\begin{axis}
	\addplot coordinates {
	(1,1.5)
	(4,16)
	(8,63.5)
	};
	\end{axis}	
	\end{tikzpicture}
	\caption{Representaci\'on puntos}
\end{wrapfigure}
\begin{verbatim}
\begin{tikzpicture}
\begin{axis}
\addplot coordinates {
	(1,1.5)
	(4,16)
	(8,63.5)
};
\end{axis}
\end{tikzpicture}
\end{verbatim}
Añadir al pre\'ambulo:
\begin{verbatim}
\usepackage{pgfplots}
\pgfplotsset{width=6cm,compat=1.12}
\end{verbatim}
\end{frame}

%%%%%%%%%%%%%%%%%%%%%%%%%%%%%%%%%%%%%%%%%%%%%
\begin{frame}[fragile]{Representaci\'on de funciones en TikZ con PGFPlots}
\begin{wrapfigure}{r}{6cm}
\begin{tikzpicture} 
\begin{axis}[ ]
% density of Normal distribution:
\addplot[red, domain=0:4,
   samples=201]
   {exp((-(x-2)^2)/2) / sqrt(2*pi)};
\end{axis}
\end{tikzpicture}
\caption{Representaci\'on funci\'on}
\end{wrapfigure}
$ $ \vspace{3cm}
\begin{verbatim}
\begin{tikzpicture} 
\begin{axis}[ ]
\addplot[ red, 
  domain=0:4,
  samples=201]
  {exp((-(x-2)^2)/2) / sqrt(2*pi)};
\end{axis}
\end{tikzpicture}
\end{verbatim}
\end{frame}

%%%%%%%%%%%%%%%%%%%%%%%%%%%%%%%%%%%%%%%%%%%%%
\begin{frame}[fragile]{Representaci\'on de funciones en TikZ con PGFPlots}
\begin{wrapfigure}{r}{5cm}
\begin{tikzpicture} 
\begin{axis}
[xlabel=Var X, ylabel=Var Y,
legend pos=north west]
\addplot[domain=0:8]{x^2};
\addplot coordinates {
	(1,1.5)
	(4,16)
	(8,63.5)
};
\legend{$f(x)=x^2$,Data}
\end{axis}
\end{tikzpicture}
\caption{Representaci\'on conjunta}
\end{wrapfigure}
$ $ %\vspace{1cm}
\begin{verbatim}
\begin{tikzpicture} 
\begin{axis}
[xlabel=Var X, ylabel=Var Y,
legend pos=north west]
\addplot[domain=0:8]{x};
\addplot coordinates {
	(1,1.5)
	(4,16)
	(8,63.5)
};
\legend{$h(x)=x$,$f(x)$}
\end{axis}
\end{tikzpicture}
\end{verbatim}
\end{frame}

%%%%%%%%%%%%%%%%%%%%%%%%%%%%%%%%%%%%%%%%%%%%%
\begin{frame}[fragile]{Representaci\'on de funciones en TikZ con PGFPlots}
\begin{wrapfigure}{r}{5cm}
\begin{tikzpicture}
\begin{axis}[
   title=\'Orbita,
   xlabel={$p(t)$},
   ylabel={$p'(t)$},
]
\addplot[blue] table 
{./graficos/type1Orbit.table};
\end{axis}
\end{tikzpicture}
\caption{Gr\'afico fichero datos}
\end{wrapfigure}
$ $ %\vspace{1cm}
\begin{verbatim}
\begin{tikzpicture}
\begin{axis}[
   title=Datos externos,
   xlabel={$x$},
   ylabel={$y$},
]
\addplot[blue] table 
{./graficos/type1Orbit.table};
\end{axis}
\end{tikzpicture}
\end{verbatim}
\end{frame}

%%%%%%%%%%%%%%%%%%%%%%%%%%%%%%%%%%%%%%%%%%%%%

\begin{frame}[fragile]{Generaci\'on de gr\'aficos en TikZ con PGFPlots}

\begin{wrapfigure}{r}{5cm}
\begin{tikzpicture} \begin{axis}[
    title={$x \exp(-x^2-y^2)$},
    xlabel=$x$, ylabel=$y$,
    small,
] \addplot3[
    surf,
    domain=-2:2,
    domain y=-1.3:1.3,
]
    {exp(-x^2-y^2)*x};
\end{axis}
\end{tikzpicture}
\caption{Gr\'afico tridimensional}
\end{wrapfigure}
$ $ %\vspace{1cm}
\begin{verbatim}
\begin{tikzpicture} 
\begin{axis}[
    title={$x \exp(-x^2-y^2)$},
    xlabel=$x$, ylabel=$y$,
    small] 
\addplot3[surf,
    domain=-2:2,
    domain y=-1.3:1.3
]{exp(-x^2-y^2)*x};
\end{axis}
\end{tikzpicture}
\end{verbatim}
\end{frame}



%%%%%%%%%%%%%%%%%%%%%%%%%%%%%
\begin{frame}[fragile]{El paquete chemfig}
Este uso se está extendiendo en \'areas distintas a la Física o las Matemática como por ejemplo el paquete \href{https://www.ctan.org/pkg/chemfig}{\color{blue} chemfig} en \href{https://www.ctan.org/search/index?phrase=chemistry&offset=48&max=16}{\color{blue}Química}
\begin{wrapfigure}{r}{4cm}
\chemfig{-(-[1]O^{\ominus})=[7]O}

\chemfig{C(-[:0]H)(-[:90]H)(-[:180]H)(-[:270]H)}
\caption{Gr\'aficos química}
\end{wrapfigure}
\begin{verbatim}
\chemfig{-(-[1]O^{\ominus})=[7]O}
\end{verbatim}
\vspace{1cm}
\begin{verbatim}
\chemfig{C(-[:0]H)(-[:90]H)
(-[:180]H)(-[:270]H)}
\end{verbatim}
\end{frame}


%%%%%%%%%%%%%%%%%%%%%%%%%%%%%%
\begin{frame}[fragile]{Funcionalidades específicas: árboles}

\begin{tikzpicture}[every node/.style={rectangle, fill=blue!20!white}]
\node {Animalia} [sibling distance=6cm]
child {node {Chordata}
	child {node {Vertebrata} [sibling distance=1.5cm]
		child {node {Mammalia}}
		child {node {Aves}}
		}
	}
child {node {Arthoropoda} [sibling distance=4cm]
	child {node {Mandibulata}[sibling distance=2cm]
		child {node {Insecta}}
		child {node {Crustacea}}
		}
	child {node {Chelicerata}
		child {node {Arachnida}}
		}
	};
\end{tikzpicture}

\href{https://www.overleaf.com/learn/latex/LaTeX_Graphics_using_TikZ%3A_A_Tutorial_for_Beginners_(Part_5)%E2%80%94Creating_Mind_Maps#:~:text=Part%204%20%7C-,Part,-5}{\textcolor{blue}{Manual: Creating Mind Maps}}
%
\end{frame}

%%%%%%%%%%%%%%%%%%%%%%%%%%%%%%
\begin{frame}[fragile]{Funcionalidades específicas: diagramas de flujo}

% Define block styles
\tikzstyle{decision} = [diamond, draw, fill=blue!20, 
    text width=3.5em, text badly centered, node distance=3cm, inner sep=0pt]
\tikzstyle{block} = [rectangle, draw, fill=blue!20, 
    text width=4em, text centered, rounded corners, minimum height=3em]
\tikzstyle{line} = [draw, -latex']
\tikzstyle{cloud} = [draw, ellipse,fill=red!20, node distance=3cm,
    minimum height=2em]
    
\begin{tikzpicture}[yscale=0.3,node distance = 1.8cm, auto]
    % Place nodes
    \node [block] (init) {\footnotesize initialize model};
    \node [cloud, left of=init] (expert) {\footnotesize expert};
    \node [cloud, right of=init] (system) {\footnotesize system};
    \node [block, below of=init] (identify) {\footnotesize identify candidate models};
    \node [block, below of=identify] (evaluate) {\footnotesize evaluate candidate models};
    \node [block, left of=evaluate, node distance=3cm] (update) {\footnotesize update model};
    \node [decision, below of=evaluate] (decide) {\footnotesize is best candidate better?};
    \node [block, right of=decide, node distance=3cm] (stop) {\footnotesize stop};
    % Draw edges
    \path [line] (init) -- (identify);
    \path [line] (identify) -- (evaluate);
    \path [line] (evaluate) -- (decide);
    \path [line] (decide) -| node [near start] {yes} (update);
    \path [line] (update) |- (identify);
    \path [line] (decide) -- node {no}(stop);
    \path [line,dashed] (expert) -- (init);
    \path [line,dashed] (system) -- (init);
    \path [line,dashed] (system) |- (evaluate);
\end{tikzpicture}
\href{https://www.overleaf.com/learn/latex/LaTeX_Graphics_using_TikZ%3A_A_Tutorial_for_Beginners_(Part_3)%E2%80%94Creating_Flowcharts#The_tikzstyle_command:~:text=Part%202%20%7C-,Part,-3%20%7C%20Part}{\textcolor{blue}{Creating flowcharts}}
\end{frame}






%%%%%%%%%%%%%%%%%%%%%%%%%%%%%%%%%%%%%%%%%%%%%
%  Código TikZ generado con otro programa
%%%%%%%%%%%%%%%%%%%%%%%%%%%%%%%%%%%%%%%%%%%%%

\section{Exportar a \LaTeX{}  desde un programa externo}

\begin{frame}{Exportar a \LaTeX{} desde un programa externo}
Debido a la progresiva implantaci\'on de \LaTeX  son  numerosos los programas que 
dan la posibilidad de exportar gr\'aficos  generados con ellos bien a TikZ
\begin{enumerate}
\item \href{https://ctan.org/pkg/gnuplottex}{\color{blue} gnuplot}
\item \href{https://mcj.sourceforge.net/latex_and_xfig.html}{\color{blue} Xfig} {\small (versi\'on para Windows: WinFIG)}, 
\item \href{http://www.geogebra.org/cms/}{\color{blue} GeoGebra}

\end{enumerate}
o bien a PSTricks
\begin{enumerate}
\item  \href{https://inkscape-manuals.readthedocs.io/en/latest/export-other-formats.html}{\color{blue} Inkscape}
\item \href{http://latexdraw.sourceforge.net}{\color{blue} LatexDraw},
\item \href{http://dia-installer.de/index.html.en}{\color{blue} Dia},
\item etc...
\end{enumerate}
\end{frame}

%%%%%%%%%%%%%%%%%%%%%%%%%%%%%%%%%%%%%%%%%%%%%
\begin{frame}{Exportar a TikZ desde Geogebra}
%\begin{exampleblock}{Ejemplo: Exportar a TIKZ  un gr\'afico con un programa externo}
\href{http://www.geogebra.org/cms/}{\color{blue} GeoGebra} es un programa especialmente sencillo para
realizar  gr\'aficos.

\definecolor{uququq}{rgb}{0.25,0.25,0.25}
\definecolor{zzttqq}{rgb}{0.6,0.2,0}
\definecolor{qqqqff}{rgb}{0,0,1}
\begin{tikzpicture}[line cap=round,line join=round,>=triangle 45,x=0.7732426303854876cm,y=0.6894409937888198cm]
\draw[->,color=black] (-4.3,0) -- (4.52,0);
\foreach \x in {-4,-3,-2,-1,1,2,3,4}
\draw[shift={(\x,0)},color=black] (0pt,2pt) -- (0pt,-2pt) node[below] {\footnotesize $\x$};
\draw[->,color=black] (0,-3.14) -- (0,3.3);
\foreach \y in {-3,-2,-1,1,2,3}
\draw[shift={(0,\y)},color=black] (2pt,0pt) -- (-2pt,0pt) node[left] {\footnotesize $\y$};
\draw[color=black] (0pt,-10pt) node[right] {\footnotesize $0$};
\clip(-4.3,-3.14) rectangle (4.52,3.3);
\onslide<1->{
  \fill[color=zzttqq,fill=zzttqq,fill opacity=0.1] (-2.16,3.2) -- (0.78,4.52) -- (3,2.68) -- (-1,1) -- cycle;
}
\onslide<1->{
  \draw [color=zzttqq] (0.78,4.52)-- (3,2.68);
}
\onslide<1->{
  \draw [color=zzttqq] (3,2.68)-- (-1,1);
}
\onslide<1->{
  \draw [color=zzttqq] (-1,1)-- (-2.16,3.2);
}
\onslide<1->{
  \draw[smooth,samples=100,domain=-4.3:4.5200000000000005] plot(\x,{sin(((\x))*180/pi)});
}
\onslide<1->{
  \draw [domain=-4.3:4.52] plot(\x,{(-0--1*\x)/1});
}
\onslide<1->{
  \draw(-3.04,4.36) ellipse (0.8cm and 0.71cm);
}
\onslide<1->{
  \draw (4.54,1.96) node[anchor=north west] {$\Sigma_{i=1}^n y_i$};
}
\begin{scriptsize}
\onslide<1->{
  \fill [color=qqqqff] (-2.16,3.2) circle (1.5pt);
}
\onslide<1->{
  \draw[color=qqqqff] (-2,3.46) node {$A$};
}
\onslide<1->{
  \fill [color=qqqqff] (0.78,4.52) circle (1.5pt);
}
\onslide<1->{
  \draw[color=qqqqff] (0.92,4.78) node {$B$};
}
\onslide<1->{
  \fill [color=qqqqff] (3,2.68) circle (1.5pt);
}
\onslide<1->{
  \draw[color=qqqqff] (3.16,2.94) node {$C$};
}
\onslide<1->{
  \fill [color=qqqqff] (-1,1) circle (1.5pt);
}
\onslide<1->{
  \draw[color=qqqqff] (-0.84,1.26) node {$D$};
}
\onslide<1->{
  \fill [color=qqqqff] (5.26,4.04) circle (1.5pt);
}
\onslide<1->{
  \draw[color=qqqqff] (5.4,4.3) node {$E$};
}
\onslide<1->{
  \fill [color=uququq] (0,0) circle (1.5pt);
}
\onslide<1->{
  \draw[color=uququq] (0.14,0.26) node {$F$};
}
\onslide<1->{
  \fill [color=qqqqff] (1,1) circle (1.5pt);
}
\onslide<1->{
  \draw[color=qqqqff] (1.16,1.26) node {$G$};
}
\onslide<1->{
  \fill [color=qqqqff] (-1.18,5.36) circle (1.5pt);
}
\onslide<1->{
  \draw[color=qqqqff] (-1.02,5.62) node {$H$};
}
\onslide<1->{
  \fill [color=qqqqff] (-3.04,4.36) circle (1.5pt);
}
\onslide<1->{
  \draw[color=qqqqff] (-2.94,4.62) node {$I$};
}
\onslide<1->{
  \fill [color=qqqqff] (-2.58,5.28) circle (1.5pt);
}
\onslide<1->{
  \draw[color=qqqqff] (-2.46,5.54) node {$J$};
}
\end{scriptsize}
\end{tikzpicture}

Nota: Ofrece la posibilidad de exportar el gr\'afico
que acabamos de realizar (puede dar errores no sencillos de detectar/corregir).
%\end{exampleblock}
\end{frame}





%%%%%%%%%%%%%%%%%%%%%%%%%%%%%%
\begin{frame}[fragile]{Uso avanzado: sitios de interés}
Puesto que es imposible mostrar paquetes de interés para una 
audiencia heterogénea lo mejor es mostrar dónde y cómo localizarlos

\href{https://www.ctan.org}{\color{blue}https://www.ctan.org}

\href{https://github.com/pgf-tikz/pgf}{\color{blue}https://github.com/pgf-tikz/pgf}

\href{https://es.wikibooks.org/wiki/Manual_de_LaTeX/Inclusión_de_gráficos/Gráficos_con_TikZ}{\color{blue}Wikibooks: Gráficos con Tikz}

\href{http://www.texample.net/tikz/examples/}{\color{blue}http://www.texample.net/tikz/examples/}

\href{https://tex.stackexchange.com/questions/42611/list-of-available-tikz-libraries-with-a-short-introduction}{\color{blue} Librerías de Tikz}

\end{frame}



\end{document}
